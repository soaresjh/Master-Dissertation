\chapter{Introduction}

Throughout the history, and prehistory, of humanity, the different eras used to be named in honor of the technology that has largely affected their behavior. Starting with the use of stones as tools to open hard shells and defend from wild animal in the Stone Age~\cite{livescience_2011}, passing through the Oil Age when the oil was discovered as a energy source~\cite{Maugeri_2007}, the Jet Age with the advance of jet-engine~\cite{brit_fly} and lots of remarkable technologies that has changed the curse of the society~\cite{brit_tec}.

Nowadays, it is notorious the amount of new technologies that rise every day; however, mostly of this technologies presents a common base, information. Our smartphones are, more than ever, presents in our life being essential, some times. Internet access is natural enough to not be noticed. More and more devices use personal, environmental and global information to provide the way of life that we know. Although a good description of our Epoch belongs to the future, it is some times called the Age of Information~\cite{TechTarget, NYT_2012, Forbes}, which is a reasonable guess. Tracing back the history we can spot the invention that lead to the rise of all the current information technology, the Optical Fiber.

The use of light to transmit information shortened the distances and extended the channel enabling to much more Data to reach much further than was possible using electronic signal. The Corning Inc., one of the greatest manufacturers of optical fiber, describe the impact of it as \textit{"revolutionized the way the World communicates. Thanks to this technology, we can connect to anyone anywhere... at the speed of light"}\footnote{Exactly comment at \url{https://www.corning.com/in/en/products/communication-networks/products/fiber/optical-fiber-innovation.html}}. However, the optical fiber is just the part of this circuit that transport the information, to produce this information, read it and apply in a useful form we need a whole range of devices, the study of this devices is a field of Photonics.

\section{Photonics}
We can understand as photonics the technology of generating and manipulation light, it can be applied in several areas such health care and life sciences, metrology, sensing, manufacturing, telecommunication, etc. We are more concerned with its role in information technology. Due to the advance in the portable devices in this area the desire for miniaturization is in constantly rise. Thanks the progress in microfabrication process and facilities its possible to fabricate micro device with specific optical characteristics that can be applied in optical circuits. One of the most common device is optical microcavities.    

The micrometric dimension of the cavity aid in the miniaturization but also lead to scenery where some physical phenomena can be easier observed. The enclose light is limited to a small volume, however the circulation power is huge increased due to photon recycling, both characteristics lead to high intensity electric field, hundreds of time higher than the intensity of usual CW laser which make optical microcavities a good platform to study effects due the interaction of the light with matter.

\section{Nonlinear Optics}
A particular way of interactions between light and matter is called Nonlinear Optics, it occurs when optical properties are affected by the presence of light. To describe this behavior we use the Nonlinear Optical Susceptibility which tell us how the presence of light affect how light propagates in that material.

The first nonlinear phenomenon was reported in 1961~\cite{Franken_1961} when it was observed the generation of second harmonic in a silica sample. Since then, the applicability of nonlinear optics has been growing and is now present in our life almost as much as lasers~\cite{Garmire_13}. 

A limitation, however, to observe nonlinear effects is the incident power. As the nonlinear optical susceptibility come from the motion of electron it is necessary an electric field comparable with the interatomic. Typically it is necessary high power laser source, as pulsed laser for example, to observe nonlinear effects~\cite{Wang_2018, Billat_2017, Giancarlo_2018}. For this reason, optical microcavities has been required in the study of nonlinear optics. Though many application of nonlinear optical effects has already been observed in microcavities\cite{Papp_14,Chen_19,Zhang_19,Li_18,Pernice_12,Luo_17,Lin_16,Li_2018}, comparation between theoretical model and experimental results yet aren't so common\cite{Khitrova_1999, Kippenberg_04}. 

In this project we will describe the theoretical model of resonant modes in microcavities and study how nonlinear optical susceptibility, more precisely third order nonlinearity, affect this modes. Moreover, we will fabricate and characterize micro wedge resonator in order to compare the theoretical results with experimental.  

\section{Structure of the Dissertation}

This Dissertation is organized in six more Chapter. 

The Chapter~\ref{chap:2_nonlin_pol} describe the dynamics of nonlinear polarization from a microscopy point of view. It starts with the interatomic potential to describe the motion of the electron when interact with a external electric field, in order to describe the origin of high harmonic terms. Them, this terms are considered in the wave equation to describe the generation of third harmonic in a bulk material.     

In Chapter~\ref{chap:3_optical_cavity} is described the behavior of confined modes. The wave equations was splitted in temporal and spatial parts that can be solved separately, in this Chapter we don't consider the effects of nonlinear polarization and describe just characteristics intrinsic of microcavities. The Chapter~\ref{chap:5_couple_mode} is answerable to merge the previous by consider the nonlinear polarization in the wave equation and solve they considering the confined boundaries conditions. 

The Chapter~\ref{chap:4_fabrication} is a \textit{Intermezzo} in the theory. Here we will describe the fabrication process to obtain the microcavities looking for details of each one of the main steps, describing they result and challenges.

The Chapter~\ref{chap:6_experiments} brings the result of the main experiments make in this project. We  started demonstrating generation of third harmonic in our devices. Next we characterize the device in the visible band using two different techniques. Then, we use computational simulation to calculate the interaction between the visible and the infrared mode. For last but not least, we tune the condition of third harmonic generation in order to compare the experimental results with the theoretical model. Finally, the Chapter~\ref{chap:7_outlook} brings the major achievement of the project as also an overview in the limitation and predictions for new studies. 