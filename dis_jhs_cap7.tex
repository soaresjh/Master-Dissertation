\chapter{Outlook}
\label{chap:7_outlook}
%We have demonstrated generation of third harmonic in optical microcavities, with a maximum net efficiency of $10^{-5}$~\cite{Soares2016}. A theoretical model for Third Harmonic Generation has been described and the experimental results was with a good agreement with the theory. 

We have described a theoretical model for Third Harmonic generation that predict a maximum efficiency conversion at a critical power. It started be describe the Single model theory for a optical cavity, then we describe the coupling between two modes assuming a third order nonlinearity. The rate equations was solved using numerical methods, later the model was compared with a Third Harmonic Generation measured in a wedge microresonator. 

The devices  used in this project was fabricated using in-house facilities. The $Q$ factor of the cavities was lower than the values reported for similar devices~\cite{Lee2012}, for that reason it was dedicated time to analyse and report details of our fabrication process which. The result of this investigation is a list of points to improve in our process that may lead us to state-of-the-art for this kind of device.

To compare the theory with the experiment, we had to measured the visible mode, which isn't usual to do in our laboratory. It was necessary to implement a new technique since we do not have a tunable visible laser. Both used techniques was in well agreement, evidencing their effectiveness. It was used a computational model to calculate the spatial distribution of the mode and infer the coupling constant. Although, the limitation on the input power prevented to reach the maximum efficiency regime.  

Improve the $Q$ factor of our device would decrease the critical power, enabling the maximum efficiency regime to be reach easier. Parallel to this, it is possible to increase the amount of collected visible light by improve the coupling between the visible mode and the waveguide, the use of a second waveguide designed for visible was already reported as a good solution~\cite{Surya17}. These would allow a impressive increase on Third Harmonic Generation efficiency~\cite{Rodriguz2007}