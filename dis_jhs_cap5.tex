\chapter{Couple Mode Theory}
\label{chap:couple_mode}
%Descrição dos modos a1 e a3 considerando a perturbação de terceira ordem no autovalor. No final da sessão vou ter descrito a teoria do mapa de eficiência. Aqui uso como gancho a necessidade de saber os valores de $\kappa_e$ e $\kappa_i$ do visível e o valor de $J_3$ como motivação para todo o resto.  

The cavities we fabricate are designed to presents multiples optical modes. In this Chapter we shall study how the nonlinearity of the material affect this modes. Initially a infrared mode are excited using a external source, the third harmonic of the source act as a new source with the triple of the frequency in the neighborhood of a visible mode, leading to a couple behavior of this modes, this coupling can be described using the rate equation  

\section{Couple Rate Equation}
%
%From now on we are interested in optical mode with specific feature. Lets assume a mode with frequency $\omega_1$ at infrared, we want to know how it couple with a visible mode of frequency $\omega_3 = 3\times\omega_1+\delta$. 
%
%\begin{subequations}
%    \begin{alignat}{2}
%        \dot{a}_1 &= -\left(i\Delta_1 + i\omega_1 A_1|a_1|^2 + i\omega_1 A_{13}|a_3|^2 + \frac{\kappa_1}{2}\right)a_1 + \sqrt{\kappa^{(e)}_1}S_{in},\\
%        \dot{a}_3 &= -\left(i\Delta_3 + i\omega_3 A_3|a_3|^2+ i\omega_3 A_{31}|a_1|^2 + \frac{\kappa_3}{2}\right)a_3.
%    \end{alignat}
%\end{subequations}

The Eq~\ref{eq:rate_equation_bistable} give us the behavior of a single mode. For problems with low nonlinearity, in other words, if we neglect terms of order $\left(\chi^{(n)}\right)^2$ and highers, the effect of the nonlinearity can be found using perturbation theory. 
\begin{equation}
    \frac{\delta\omega_\alpha}{\omega_\alpha} = \frac{1}{2}\frac{\bra{\vec{E}_\alpha}\delta\epsilon\ket{\vec{E}}}{\braket{\vec{E}_\alpha|\vec{E}_\alpha}} = \frac{1}{2}\frac{\int\vec{E}^*_\alpha\delta\vec{P}d^3\vec{r}}{\int \epsilon|\vec{E}_\alpha|^2d^3\vec{r}}
    \label{eq:perp_theory}
\end{equation}
Note that $\vec{E}_\alpha$ is the unperturbed electric field of the $\alpha$th mode, while $\vec{E}$ is the total electric field. 

For a generic problem $\delta\epsilon\ket{\vec{E}}=\ket{\delta\vec{P}}$ is the change in the polarization due to nonlinearity perturbation, as we are studding third order nonlinearity this perturbation can be write as $\delta\vec{P} = \epsilon\chi^{(3)}|\vec{E}|^2\vec{E}$. Our model assumes that there is only two different modes coupled, in such way that, the total electric field is
\begin{equation}
    \vec{E} = \vec{E}_1+\vec{E}_3+c.c.
\end{equation}
From now on, the label $1$ refers to the model in the infrared and the label $3$ to the mode in the visible. 
The calculus of the Eq~\ref{eq:perp_theory} give rise to a total of sixty four terms for each one of the modes, however most of then are not of our interest. In order to filter this terms we will write the electric field with a explicit dependency of time $\vec{E}_\alpha(\vec{r},t) = a_\alpha(t) \vec{e}_\alpha(\vec{r})$. Considering the approximation of slowly variation, the amplitude $a_\alpha(t)$ can be considered as a harmonic function with frequency $\omega_\alpha$. Applying this trick, some terms of the Eq~\ref{eq:perp_theory} can be considered out of frequency and be neglected. The full expression for the perturbation for each mode is

\begin{eqnarray}
\frac{\delta\omega_1}{\omega_1} &=& -\frac{1}{8}\left[|a_1|^2\frac{\int\epsilon\chi^{(3)}
\Big(|\vec{e}_1\cdot\vec{e}_1|^2 + 2|\vec{e}_1\cdot\vec{e}_1^*|^2
\Big)d^3\vec{r}}
{\int \epsilon|\vec{e}_1|^2 d^3\vec{r}}\right. +\nonumber\\
&+&2|a_3|^2\frac{\int\epsilon\chi^{(3)}
\Big((\vec{e}_1\cdot\vec{e}_1^*)(\vec{e}_3\cdot\vec{e}_3^*)+|\vec{e}_1\cdot\vec{e}_3|^2+|\vec{e}_1\cdot\vec{e}^*_3|^2
\Big)d^3\vec{r}}
{\int \epsilon|\vec{e}_1|^2 d^3\vec{r}}+\nonumber\\
&+&\left.\frac{(a^*_1)^2a_3}{a_1}\frac{\int\epsilon\chi^{(3)}
\Big(3(\vec{e}^*_1\cdot\vec{e}_1^*)(\vec{e}^*_1\cdot\vec{e}_3)
\Big)d^3\vec{r}}
{\int \epsilon|\vec{e}_1|^2 d^3\vec{r}}\right]
\end{eqnarray}

\begin{eqnarray}
\frac{\delta\omega_3}{\omega_3} &=& -\frac{1}{8}\left[|a_3|^2\frac{\int\epsilon\chi^{(3)}
\Big(|\vec{e}_3\cdot\vec{e}_3|^2 + 2|\vec{e}_3\cdot\vec{e}_3^*|^2
\Big)d^3\vec{r}}
{\int \epsilon|\vec{e}_3|^2 d^3\vec{r}}\right. +\nonumber\\
&+&2|a_1|^2\frac{\int\epsilon\chi^{(3)}
\Big((\vec{e}_1\cdot\vec{e}_1^*)(\vec{e}_3\cdot\vec{e}_3^*)+|\vec{e}_1\cdot\vec{e}_3|^2+|\vec{e}_1\cdot\vec{e}^*_3|^2
\Big)d^3\vec{r}}
{\int \epsilon|\vec{e}_3|^2 d^3\vec{r}}+\nonumber\\
&+&\left.\frac{(a_1)^3}{a_3}\frac{\int\epsilon\chi^{(3)}
\Big(3(\vec{e}_1\cdot\vec{e}_1)(\vec{e}_1\cdot\vec{e}^*_3)
\Big)d^3\vec{r}}
{\int \epsilon|\vec{e}_3|^2 d^3\vec{r}}\right]
\end{eqnarray}

In order to normalize the amplitude in such way that $|a_\alpha(t)|^2$ is the energy stored in the $\alpha$th mode, we should normalize the electric field $\vec{e}_\alpha \rightarrow\vec{e}_\alpha/\int|\vec{e}_\alpha|^2 d^3\vec{r}$. Using this form, the perturbation can be linked with the rate equation making $\omega_\alpha \rightarrow \omega_\alpha + \delta\omega_\alpha$. We can factorize the expression based in the dependence of the amplitude. 

\begin{eqnarray}
\frac{\delta\omega_1}{\omega_1} &=& A_{11}|a_1|^2 + A_{13}|a_3|^2 + B_1\frac{(a^*_1)^2a_3}{a_1},\label{eq:ir_per}\\
\frac{\delta\omega_3}{\omega_3} &=& A_{33}|a_3|^2 + A_{31}|a_1|^2 + B_3\frac{(a_1)^3}{a_3}\label{eq:vis_per}.
\end{eqnarray}

%The rate equation for both modes are write as
Once identified all terms, we use the rotation frame approximation making $a_1 \rightarrow a_1e^{i\omega t}$ and $a_3 \rightarrow a_3e^{i3\omega t}$, with $\omega$ being the source frequency. Them we have 
\begin{subequations}
    \begin{alignat}{1}
        \dot{a}_1 &= -\left(i\Delta_1 + \omega_1i(A_{11} |a_1|^2 + A_{13} |a_3|^2) + \kappa_{1}\right)a_1 - i\omega_1 B_1(a^*_1)^2a_3 
        +\sqrt{2 \kappa^{(e)}_1}S_{in}
        \label{eq:taxa_ir_broad}\\
        \dot{a}_3 &= -\left(i\Delta_3 + \omega_3i(A_{31} |a_1|^2 + A_{33} |a_3|^2) + \kappa_{3}\right)a_3 - i\omega_3 B_3(a_1)^3.
        \label{eq:taxa_vis_broad}
    \end{alignat}
    \label{eq:rate_broad}
\end{subequations}
This move lead us to few new terms. Lets take a look to each one of them. Initially, we have $\Delta_1 = \omega_1 - \omega$ and $\Delta_3 = \omega_3 - 3\times\omega$. 

The terms $A_{\alpha\alpha}$ look like to the SPM term presented in Eq~\ref{eq:rate_equation_bistable}, in fact both are responsible for the same effect. 

The directly dependence of the refractive index with the intensity of the electric field is called the Optical Kerr effect. It is possible to show that due to third order nonlinearity, the refractive index can be write as~\needcit
\begin{equation}
    \text{n} = \text{n}_0 + 2\bar{\text{n}}_2|E|^2 
\end{equation}
If included in the wave equation, it lead to the same bistable behavior discussed before. The difference between Kerr bistability and thermotic bistability lies in the nature of both. The Kerr effect responds at a time scale faster than thermotic~\needcit and have to due just with the material, meanwhile the thermotic bistability is a intrinsic behavior of the optical cavity and can be affect by both, material and geometry. However, at the specific experimental condition, we can not distinguish the individual contribution of each one in the transmission~\needcit, hence we consider just one SPM term that includes contributions of both, Kerr and thermotic. 

Futhermore, the Eqs~\ref{eq:rate_broad} present another similar term, with the form $A_{ij}|a_j|^2$. The effec of this term is exactly the one discussed before, but the bistability are generated due to the other optical mode, for both Kerr and thermotic~\needcit. This term are called Cross Phase Modulation (XPM). The expression for Kerr SPM and XPM can be take from the perturbation theory as 


\begin{subequations}
    \begin{alignat}{1}
        A_{ii} &= \frac{1}{8}\frac{\int\epsilon\chi^{(3)}
        \Big[|\vec{e}_i\cdot\vec{e}_i|^2 + 2|\vec{e}_i\cdot\vec{e}_i^*|^2
        \Big]d^3\vec{r}}{\Big[\int \epsilon|\vec{e}_i|^2 d^3\vec{r}\Big]^2},
        \\
        A_{ji} &= \frac{1}{4}\frac{\int\epsilon\chi^{(3)}
        \Big[|\vec{e}_i|^2|\vec{e}_j|^2 + |\vec{e}_i\cdot\vec{e}_j|^2+ |\vec{e}_i\cdot\vec{e}_j^*|^2
        \Big]d^3\vec{r}}{\Big[\int \epsilon|\vec{e}_i|^2 d^3\vec{r}\Big]\Big[\int \epsilon|\vec{e}_j|^2 d^3\vec{r}\Big]}.
    \end{alignat}
\end{subequations}

The other new terms are thus with $B_1$ and $B_3$. It is important to emphasize the fact that the amplitude $a_\omega$ is complex, so this term presents a real and a imaginary part, which act as a source term. We will call it the Coupling term, it is responsible for convert energy from one mode to the other. The expression for both, can be take from the perturbation theory as
\begin{subequations}
    \begin{alignat}{1}
        B_{1} &= \frac{3}{8}\frac{\int\epsilon\chi^{(3)}
        (\vec{e}_1^*\cdot\vec{e}_1^*)(\vec{e}_1^*\cdot\vec{e}_3)
        d^3\vec{r}}{\Big[\int \epsilon|\vec{e}_1|^2 d^3\vec{r}\Big]^{3/2}\Big[\int \epsilon|\vec{e}_3|^2 d^3\vec{r}\Big]^{1/2}},
        \label{eq:coupling_b1}
        \\
        B_{3} &= \frac{1}{8}\frac{\int\epsilon\chi^{(3)}
        (\vec{e}_1\cdot\vec{e}_1)(\vec{e}_1\cdot\vec{e}^*_3)
        d^3\vec{r}}{\Big[\int \epsilon|\vec{e}_1|^2 d^3\vec{r}\Big]^{3/2}\Big[\int \epsilon|\vec{e}_3|^2 d^3\vec{r}\Big]^{1/2}}.
        \label{eq:coupling_b3}
    \end{alignat}
    \label{eq:coupling}
\end{subequations}

%Using energy conservation $\frac{d}{dt}\left(|a_1|^2+|a_3|^2\right) = 0$ we can find that $\omega_1B_1 = \omega_3B^*_3$ 
In order to solve the Coupled Rate Equations Eq~\ref{eq:rate_broad} we will use a numerical method with the software Mathematica, the Appendix~\ref{app:numerical_met} treat about the method, now we are interested in the results. 

\section{Numerical Results}

Our problem can be modulate as show in Fig~%\ref
A system compound by a cavity with two modes, one in infrared $a_1$ and one in visible $a_3$, a input wave $S_{in}$ with frequency near of the infrared mode. Both modes couple with the source, the Output wave are defined as 
\begin{subequations}
    \begin{align}
        S_1 &= S_{in} - \sqrt{\kappa^{(e)}_1}a_1\\
        S_3 &= \sqrt{\kappa^{(e)}_3}a_3
    \end{align}
\end{subequations}

Initially, lest solve the Eq~\ref{eq:rate_broad} assuming that $\omega_3 = 3\times\omega_1$. The Fig~%\ref
brings a well know bistable curve for the infrared mode (Red) while the Fig~% 
shows a slightly deformed curve for the visible mode. At first look the deformation can be explained by the cubic relation between the third harmonic power and the infrared power, although this result are true just if we assume that $\omega_3 = 3\times\omega_1$, which is not true due to the dispersion of the modes. A more general model assume that $\omega_3 = 3\times\omega_1 + \delta\omega$, in such case, the shape of the curve can be totally changed in function of the value of $\delta\omega$. The Fig~%\
shows some case. 

This dynamics occurs due to the SPM and XPM terms. The displacement from the initial frequency are different for each mode 